% $Id: root.tex,v 1.2 2007-11-12 21:00:45 nipkow Exp $

\documentclass[11pt,a4paper,parskip,abstract]{scrartcl}
\usepackage{isabelle,isabellesym}

\usepackage[normalem]{ulem}

% further packages required for unusual symbols (see also isabellesym.sty)
% use only when needed
\usepackage{amssymb}                  % for \<leadsto>, \<box>, \<diamond>,
                                       % \<sqsupset>, \<mho>, \<Join>, 
                                       % \<lhd>, \<lesssim>, \<greatersim>,
                                       % \<lessapprox>, \<greaterapprox>,
                                       % \<triangleq>, \<yen>, \<lozenge>
%\usepackage[greek,english]{babel}     % greek for \<euro>,
                                       % english for \<guillemotleft>, 
                                       %             \<guillemotright>
                                       % default language = last
%\usepackage[latin1]{inputenc}         % for \<onesuperior>, \<onequarter>,
                                       % \<twosuperior>, \<onehalf>,
                                       % \<threesuperior>, \<threequarters>,
                                       % \<degree>
%\usepackage[only,bigsqcap]{stmaryrd}  % for \<Sqinter>
%\usepackage{eufrak}                   % for \<AA> ... \<ZZ>, \<aa> ... \<zz>
                                       % (only needed if amssymb not used)
%\usepackage{textcomp}                 % for \<cent>, \<currency>

\usepackage{amsmath}

% this should be the last package used
\usepackage{pdfsetup}

% urls in roman style, theory text in math-similar italics
\urlstyle{rm}
\isabellestyle{it}

\newcommand{\isasymbinit}{\isamath{b_0}}
\newcommand{\isasymabinit}{\isamath{\widehat{b_0}}}
\newcommand{\isasymPR}{\isamath{\mathcal{PR}}}
\newcommand{\isasymaPR}{\isamath{\widehat{\mathcal{PR}}}}
\newcommand{\isasymanb}{\isamath{\widehat{{nb}}}}
\newcommand{\isasymaA}{\isamath{\widehat{\mathcal{A}}}}
\newcommand{\isasymaF}{\isamath{\widehat{\mathcal{F}}}}
\newcommand{\isasymaC}{\isamath{\widehat{\mathcal{C}}}}
% Types
\newcommand{\isasymabenv}{\isamath{\widehat{{benv}}}}
\newcommand{\isasymavenv}{\isamath{\widehat{{venv}}}}
\newcommand{\isasymaclosure}{\isamath{\widehat{{closure}}}}
\newcommand{\isasymaproc}{\isamath{\widehat{{proc}}}}
\newcommand{\isasymad}{\isamath{\widehat{{d}}}}
\newcommand{\isasymafstate}{\isamath{\widehat{{fstate}}}}
\newcommand{\isasymacstate}{\isamath{\widehat{{cstate}}}}
\newcommand{\isasymaccache}{\isamath{\widehat{{ccache}}}}
\newcommand{\isasymaans}{\isamath{\widehat{{ans}}}}

\newcommand{\isactrlps}[1]{{\uline #1}}


\begin{document}

\title{Control Flow Graph}
\author{Joachim Breitner}
\maketitle

\begin{abstract}
  In his dissertation\cite{Shivers}, Olin Shivers introduces a concept of control flow graphs
  for functional languages, provides an algorithm to statically derive a safe
  approximation of the control flow graph and proves this algorithm correct. In
  this student research project, Shiver’s algorithms and proofs are formalized
  using the theorem prover system Isabelle.
\end{abstract}

\tableofcontents

% include generated text of all theories
%\input{session}

\section{Introduction}

Blubb blah

\part{The definitions}
%
\begin{isabellebody}%
\def\isabellecontext{CPSScheme}%
%
\isamarkuptrue%
\isacommand{types}\isamarkupfalse%
\ label\ {\isacharequal}\ nat\isanewline
\isanewline
\isacommand{types}\isamarkupfalse%
\ var\ {\isacharequal}\ {\isachardoublequoteopen}label\ {\isasymtimes}\ string{\isachardoublequoteclose}\isanewline
\isanewline
\isamarkuptrue%
\isacommand{datatype}\isamarkupfalse%
\ prim\ {\isacharequal}\ Plus\ label\ {\isacharbar}\ If\ label\ label\isanewline
\isanewline
\isacommand{datatype}\isamarkupfalse%
\ lambda\ {\isacharequal}\ Lambda\ label\ {\isachardoublequoteopen}var\ list{\isachardoublequoteclose}\ call\isanewline
\ \ \ \ \ \isakeyword{and}\ call\ {\isacharequal}\ App\ label\ val\ {\isachardoublequoteopen}val\ list{\isachardoublequoteclose}\isanewline
\ \ \ \ \ \ \ \ \ \ \ \ \ \ {\isacharbar}\ Let\ label\ {\isachardoublequoteopen}{\isacharparenleft}var\ {\isasymtimes}\ lambda{\isacharparenright}\ list{\isachardoublequoteclose}\ call\isanewline
\ \ \ \ \ \isakeyword{and}\ val\ {\isacharequal}\ L\ lambda\ {\isacharbar}\ R\ label\ var\ {\isacharbar}\ C\ label\ int\ {\isacharbar}\ P\ prim\isanewline
\isanewline
\isacommand{types}\isamarkupfalse%
\ prog\ {\isacharequal}\ lambda%
\end{isabellebody}%
%%% Local Variables:
%%% mode: latex
%%% TeX-master: "root"
%%% End:

\input{Eval}
\input{ExCF}
\input{AbsCF}

\part{The main results}
%
\begin{isabellebody}%
\def\isabellecontext{ExCFSV}%
\isamarkuptrue%
\isacommand{lemma}\isamarkupfalse%
\ cc{\isacharunderscore}single{\isacharunderscore}valued{\isacharprime}{\isacharcolon}\isanewline
\ \ \ \ \ \ {\isachardoublequoteopen}{\isasymlbrakk}\ {\isasymforall}b{\isacharprime}\ {\isasymin}\ contours{\isacharunderscore}in{\isacharunderscore}ve\ ve{\isachardot}\ b{\isacharprime}\ {\isacharless}\ b\isanewline
\ \ \ \ \ \ \ {\isacharsemicolon}\ {\isasymforall}b{\isacharprime}\ {\isasymin}\ contours{\isacharunderscore}in{\isacharunderscore}d\ d{\isachardot}\ b{\isacharprime}\ {\isacharless}\ b\isanewline
\ \ \ \ \ \ \ {\isacharsemicolon}\ {\isasymforall}d{\isacharprime}\ {\isasymin}\ set\ ds{\isachardot}\ {\isasymforall}b{\isacharprime}\ {\isasymin}\ contours{\isacharunderscore}in{\isacharunderscore}d\ d{\isacharprime}{\isachardot}\ b{\isacharprime}\ {\isacharless}\ b\isanewline
\ \ \ \ \ \ \ {\isasymrbrakk}\isanewline
\ \ \ \ \ \ \ {\isasymLongrightarrow}\isanewline
\ \ \ \ \ \ \ {\isacharparenleft}\ \ \ single{\isacharunderscore}valued\ {\isacharparenleft}{\isasymF}{\isasymcdot}{\isacharparenleft}Discr\ {\isacharparenleft}d{\isacharcomma}ds{\isacharcomma}ve{\isacharcomma}b{\isacharparenright}{\isacharparenright}{\isacharparenright}\isanewline
\ \ \ \ \ \ \ {\isasymand}\ {\isacharparenleft}{\isasymforall}\ {\isacharparenleft}{\isacharparenleft}lab{\isacharcomma}{\isasymbeta}{\isacharparenright}{\isacharcomma}t{\isacharparenright}\ {\isasymin}\ {\isasymF}{\isasymcdot}{\isacharparenleft}Discr\ {\isacharparenleft}d{\isacharcomma}ds{\isacharcomma}ve{\isacharcomma}\ b{\isacharparenright}{\isacharparenright}{\isachardot}\isanewline
\ \ \ \ \ \ \ \ \ \ \ \ \ \ \ {\isasymexists}\ b{\isacharprime}{\isachardot}\ b{\isacharprime}\ {\isasymin}\ ran\ {\isasymbeta}\ {\isasymand}\ b\ {\isasymle}\ b{\isacharprime}{\isacharparenright}\isanewline
\ \ \ \ \ \ \ {\isacharparenright}{\isachardoublequoteclose}\isanewline
\ \ \isakeyword{and}\ {\isachardoublequoteopen}{\isasymlbrakk}\ b\ {\isasymin}\ ran\ {\isasymbeta}{\isacharprime}\isanewline
\ \ \ \ \ \ \ {\isacharsemicolon}\ {\isasymforall}b{\isacharprime}{\isasymin}ran\ {\isasymbeta}{\isacharprime}{\isachardot}\ b{\isacharprime}\ {\isasymle}\ b\isanewline
\ \ \ \ \ \ \ {\isacharsemicolon}\ {\isasymforall}b{\isacharprime}\ {\isasymin}\ contours{\isacharunderscore}in{\isacharunderscore}ve\ ve{\isachardot}\ b{\isacharprime}\ {\isasymle}\ b\isanewline
\ \ \ \ \ \ \ {\isasymrbrakk}\isanewline
\ \ \ \ \ \ \ {\isasymLongrightarrow}\isanewline
\ \ \ \ \ \ \ {\isacharparenleft}\ \ \ single{\isacharunderscore}valued\ {\isacharparenleft}{\isasymC}{\isasymcdot}{\isacharparenleft}Discr\ {\isacharparenleft}c{\isacharcomma}{\isasymbeta}{\isacharprime}{\isacharcomma}ve{\isacharcomma}b{\isacharparenright}{\isacharparenright}{\isacharparenright}\isanewline
\ \ \ \ \ \ \ {\isasymand}\ {\isacharparenleft}{\isasymforall}\ {\isacharparenleft}{\isacharparenleft}lab{\isacharcomma}{\isasymbeta}{\isacharparenright}{\isacharcomma}t{\isacharparenright}\ {\isasymin}\ {\isasymC}{\isasymcdot}{\isacharparenleft}Discr\ {\isacharparenleft}c{\isacharcomma}{\isasymbeta}{\isacharprime}{\isacharcomma}ve{\isacharcomma}b{\isacharparenright}{\isacharparenright}{\isachardot}\isanewline
\ \ \ \ \ \ \ \ \ \ \ \ \ \ \ {\isasymexists}\ b{\isacharprime}{\isachardot}\ b{\isacharprime}\ {\isasymin}\ ran\ {\isasymbeta}\ {\isasymand}\ b\ {\isasymle}\ b{\isacharprime}{\isacharparenright}\isanewline
\ \ \ \ \ \ \ {\isacharparenright}{\isachardoublequoteclose}\isanewline
%
\isadelimproof
%
\endisadelimproof
\isanewline
\isacommand{lemma}\isamarkupfalse%
\ evalPR{\isacharunderscore}single{\isacharunderscore}valued{\isacharcolon}\isanewline%
\ \ \ \  {\isachardoublequoteopen}single{\isacharunderscore}valued\ {\isacharparenleft}{\isasymPR}\ prog{\isacharparenright}{\isachardoublequoteclose}
\end{isabellebody}%
%%% Local Variables:
%%% mode: latex
%%% TeX-master: "root"
%%% End:

%
\begin{isabellebody}%
\def\isabellecontext{AbsCFCorrect}%
\isamarkuptrue%
\isacommand{lemma}\isamarkupfalse%
\ lemma{\isadigit{8}}{\isadigit{9}}{\isacharcolon}\isanewline
\ \isakeyword{fixes}\ fstate{\isacharunderscore}a\ {\isacharcolon}{\isacharcolon}\ {\isachardoublequoteopen}{\isacharprime}c{\isacharcolon}{\isacharcolon}contour{\isacharunderscore}a\ {\isasymafstate}{\isachardoublequoteclose}\ \isakeyword{and}\ cstate{\isacharunderscore}a\ {\isacharcolon}{\isacharcolon}\ {\isachardoublequoteopen}{\isacharprime}c{\isacharcolon}{\isacharcolon}contour{\isacharunderscore}a\ {\isasymacstate}{\isachardoublequoteclose}\isanewline
\ \isakeyword{shows}\ {\isachardoublequoteopen}{\isacharbar}fstate{\isacharbar}\ {\isasymlessapprox}\ fstate{\isacharunderscore}a\ {\isasymLongrightarrow}\ {\isacharbar}{\isasymF}{\isasymcdot}{\isacharparenleft}Discr\ fstate{\isacharparenright}{\isacharbar}\ {\isasymlessapprox}\ {\isasymaF}{\isasymcdot}{\isacharparenleft}Discr\ fstate{\isacharunderscore}a{\isacharparenright}{\isachardoublequoteclose}\isanewline
\ \ \ \isakeyword{and}\ {\isachardoublequoteopen}{\isacharbar}cstate{\isacharbar}\ {\isasymlessapprox}\ cstate{\isacharunderscore}a\ {\isasymLongrightarrow}\ {\isacharbar}{\isasymC}{\isasymcdot}{\isacharparenleft}Discr\ cstate{\isacharparenright}{\isacharbar}\ {\isasymlessapprox}\ {\isasymaC}{\isasymcdot}{\isacharparenleft}Discr\ cstate{\isacharunderscore}a{\isacharparenright}{\isachardoublequoteclose}\isanewline
%
\isanewline%
\isamarkuptrue%
\isacommand{lemma}\isamarkupfalse%
\ lemma{\isadigit{6}}{\isacharcolon}\ {\isachardoublequoteopen}{\isacharbar}{\isasymPR}\ l{\isacharbar}\ {\isasymlessapprox}\ {\isasymaPR}\ l{\isachardoublequoteclose}
\end{isabellebody}%
%%% Local Variables:
%%% mode: latex
%%% TeX-master: "root"
%%% End:

\input{Computability}
\input{AbsCFComp}

\part{The auxillary theories}
%
\begin{isabellebody}%
\def\isabellecontext{CPSUtils}%
%
\isacommand{lemma}\isamarkupfalse%
\ \isanewline
\ \ \isakeyword{shows}\ lambdas{\isadigit{1}}{\isacharcolon}\ {\isachardoublequoteopen}Lambda\ l\ vs\ c\ {\isasymin}\ lambdas\ x\ {\isasymLongrightarrow}\ c\ {\isasymin}\ calls\ x{\isachardoublequoteclose}\isanewline
\ \ \isakeyword{and}\ {\isachardoublequoteopen}Lambda\ l\ vs\ c\ {\isasymin}\ lambdasC\ y\ {\isasymLongrightarrow}\ c\ {\isasymin}\ callsC\ y{\isachardoublequoteclose}\isanewline
\ \ \isakeyword{and}\ {\isachardoublequoteopen}Lambda\ l\ vs\ c\ {\isasymin}\ lambdasV\ z\ {\isasymLongrightarrow}\ c\ {\isasymin}\ callsV\ z{\isachardoublequoteclose}\isanewline
\ \ \isakeyword{and}\ {\isachardoublequoteopen}{\isasymforall}z{\isasymin}\ set\ list{\isachardot}\ Lambda\ l\ vs\ c\ {\isasymin}\ lambdasV\ z\ {\isasymlongrightarrow}\ c\ {\isasymin}\ callsV\ z{\isachardoublequoteclose}\isanewline
\ \ \isakeyword{and}\ {\isachardoublequoteopen}{\isasymforall}x{\isasymin}\ set\ {\isacharparenleft}list{\isadigit{2}}\ {\isacharcolon}{\isacharcolon}\ {\isacharparenleft}var\ {\isasymtimes}\ lambda{\isacharparenright}\ list{\isacharparenright}\ {\isachardot}\ Lambda\ l\ vs\ c\ {\isasymin}\ lambdas\ {\isacharparenleft}snd\ x{\isacharparenright}\ {\isasymlongrightarrow}\ c\ {\isasymin}\ calls\ {\isacharparenleft}snd\ x{\isacharparenright}{\isachardoublequoteclose}\isanewline
\ \ \isakeyword{and}\ {\isachardoublequoteopen}Lambda\ l\ vs\ c\ {\isasymin}\ lambdas\ {\isacharparenleft}snd\ {\isacharparenleft}t{\isacharcolon}{\isacharcolon}\ var{\isasymtimes}lambda{\isacharparenright}{\isacharparenright}\ {\isasymLongrightarrow}\ c\ {\isasymin}\ calls\ {\isacharparenleft}snd\ t{\isacharparenright}{\isachardoublequoteclose}\isanewline
%
\isadelimproof
%
\endisadelimproof
%
\isatagproof
\isacommand{apply}\isamarkupfalse%
\ {\isacharparenleft}induct\ rule{\isacharcolon}lambda{\isacharunderscore}call{\isacharunderscore}val{\isachardot}inducts{\isacharparenright}\isanewline
\isacommand{apply}\isamarkupfalse%
\ auto\isanewline
\isacommand{apply}\isamarkupfalse%
\ {\isacharparenleft}case{\isacharunderscore}tac\ c{\isacharcomma}\ auto{\isacharparenright}{\isacharbrackleft}{\isadigit{1}}{\isacharbrackright}\isanewline
\isacommand{apply}\isamarkupfalse%
\ {\isacharparenleft}rule{\isacharunderscore}tac\ x{\isacharequal}{\isachardoublequoteopen}{\isacharparenleft}{\isacharparenleft}a{\isacharcomma}\ b{\isacharparenright}{\isacharcomma}\ ba{\isacharparenright}{\isachardoublequoteclose}\ \isakeyword{in}\ bexI{\isacharcomma}\ auto{\isacharparenright}\isanewline
\isacommand{done}\isamarkupfalse%
%
\endisatagproof
{\isafoldproof}%
\end{isabellebody}%
%%% Local Variables:
%%% mode: latex
%%% TeX-master: "root"
%%% End:

\input{Utils}
\input{SetMap}
\input{MapSets}
\input{HOLCFUtils}
\input{FixTransform}

\bibliographystyle{abbrv}
\bibliography{root}

\end{document}
