% $Id: root.tex,v 1.2 2007-11-12 21:00:45 nipkow Exp $

\documentclass[11pt,a4paper]{article}
\usepackage{isabelle,isabellesym}

\usepackage[normalem]{ulem}

% further packages required for unusual symbols (see also isabellesym.sty)
% use only when needed
\usepackage{amssymb}                  % for \<leadsto>, \<box>, \<diamond>,
                                       % \<sqsupset>, \<mho>, \<Join>, 
                                       % \<lhd>, \<lesssim>, \<greatersim>,
                                       % \<lessapprox>, \<greaterapprox>,
                                       % \<triangleq>, \<yen>, \<lozenge>
%\usepackage[greek,english]{babel}     % greek for \<euro>,
                                       % english for \<guillemotleft>, 
                                       %             \<guillemotright>
                                       % default language = last
%\usepackage[latin1]{inputenc}         % for \<onesuperior>, \<onequarter>,
                                       % \<twosuperior>, \<onehalf>,
                                       % \<threesuperior>, \<threequarters>,
                                       % \<degree>
%\usepackage[only,bigsqcap]{stmaryrd}  % for \<Sqinter>
%\usepackage{eufrak}                   % for \<AA> ... \<ZZ>, \<aa> ... \<zz>
                                       % (only needed if amssymb not used)
%\usepackage{textcomp}                 % for \<cent>, \<currency>

% this should be the last package used
\usepackage{pdfsetup}

% urls in roman style, theory text in math-similar italics
\urlstyle{rm}
\isabellestyle{it}

\newcommand{\isasymbinit}{\isamath{b_0}}
\newcommand{\isasymabinit}{\isamath{\widehat{b_0}}}
\newcommand{\isasymPR}{\isamath{\mathcal{PR}}}
\newcommand{\isasymaPR}{\isamath{\widehat{\mathcal{PR}}}}
\newcommand{\isasymanb}{\isamath{\widehat{{nb}}}}
\newcommand{\isasymaA}{\isamath{\widehat{\mathcal{A}}}}
\newcommand{\isasymaF}{\isamath{\widehat{\mathcal{F}}}}
\newcommand{\isasymaC}{\isamath{\widehat{\mathcal{C}}}}
% Types
\newcommand{\isasymabenv}{\isamath{\widehat{{benv}}}}
\newcommand{\isasymavenv}{\isamath{\widehat{{venv}}}}
\newcommand{\isasymaclosure}{\isamath{\widehat{{closure}}}}
\newcommand{\isasymaproc}{\isamath{\widehat{{proc}}}}
\newcommand{\isasymad}{\isamath{\widehat{{d}}}}
\newcommand{\isasymafstate}{\isamath{\widehat{{fstate}}}}
\newcommand{\isasymacstate}{\isamath{\widehat{{cstate}}}}
\newcommand{\isasymaccache}{\isamath{\widehat{{ccache}}}}
\newcommand{\isasymaans}{\isamath{\widehat{{ans}}}}

\newcommand{\isactrlps}[1]{{\uline #1}}


\begin{document}

\title{Control Flow Graph}
\author{Joachim Breitner}
\maketitle

\begin{abstract}
  In his dissertation\cite{Shivers}, Olin Shivers introduces a concept of control flow graphs
  for functional languages, provides an algorithm to statically derive a safe
  approximation of the control flow graph and proves this algorithm correct. In
  this student research project, Shiver’s algorithms and proofs are formalized
  using the theorem prover system Isabelle.
\end{abstract}

\tableofcontents

% include generated text of all theories
%\input{session}

\section{Introduction}

Blubb blah

%
\begin{isabellebody}%
\def\isabellecontext{CPSScheme}%
%
\isamarkuptrue%
\isacommand{types}\isamarkupfalse%
\ label\ {\isacharequal}\ nat\isanewline
\isanewline
\isacommand{types}\isamarkupfalse%
\ var\ {\isacharequal}\ {\isachardoublequoteopen}label\ {\isasymtimes}\ string{\isachardoublequoteclose}\isanewline
\isanewline
\isamarkuptrue%
\isacommand{datatype}\isamarkupfalse%
\ prim\ {\isacharequal}\ Plus\ label\ {\isacharbar}\ If\ label\ label\isanewline
\isanewline
\isacommand{datatype}\isamarkupfalse%
\ lambda\ {\isacharequal}\ Lambda\ label\ {\isachardoublequoteopen}var\ list{\isachardoublequoteclose}\ call\isanewline
\ \ \ \ \ \isakeyword{and}\ call\ {\isacharequal}\ App\ label\ val\ {\isachardoublequoteopen}val\ list{\isachardoublequoteclose}\isanewline
\ \ \ \ \ \ \ \ \ \ \ \ \ \ {\isacharbar}\ Let\ label\ {\isachardoublequoteopen}{\isacharparenleft}var\ {\isasymtimes}\ lambda{\isacharparenright}\ list{\isachardoublequoteclose}\ call\isanewline
\ \ \ \ \ \isakeyword{and}\ val\ {\isacharequal}\ L\ lambda\ {\isacharbar}\ R\ label\ var\ {\isacharbar}\ C\ label\ int\ {\isacharbar}\ P\ prim\isanewline
\isanewline
\isacommand{types}\isamarkupfalse%
\ prog\ {\isacharequal}\ lambda%
\end{isabellebody}%
%%% Local Variables:
%%% mode: latex
%%% TeX-master: "root"
%%% End:

\input{Eval}
\input{ExCF}

\bibliographystyle{abbrv}
\bibliography{root}

\end{document}
