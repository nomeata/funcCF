\documentclass{beamer}
%\documentclass{article}
%\usepackage{beamerarticle}

\newlength{\px}
\setlength{\px}{0.0009765625\paperwidth}

\usetheme[footline=authortitle,compress]{Nomeata}

\AtBeginSection[]
{
%  \begin{frame}<beamer>
%    \tableofcontents[currentsection,currentsubsection]
%  \end{frame}
}


%\usecolortheme{orchid}
\usecolortheme{crane}

\usepackage[T3,OT1]{fontenc}
\usepackage[german]{babel}
\usepackage[utf8]{inputenc}
\usepackage{tikz}
%\usetikzlibrary{shapes,arrows}
\usetikzlibrary{positioning,calc,decorations,decorations.pathmorphing,shapes.geometric,matrix}
\usepackage{hyperref}

\usepackage{mathtools}
\usepackage{amssymb}

\usepackage{listings}

% \pause mit verstecken
\newcommand{\hide}{\onslide+<+(1)->}

\title{Shivers Steuerungsflussanalyse}
\subtitle{Vortrag zur Studienarbeit}
\author{Joachim Breitner}
%\institute{\url{mail@joachim-breitner.de}}
%\titlegraphic{}
\date{23. November 2010}

\DeclareTextSymbol\textlambda{T3}{171}           % Lambda
\DeclareTextSymbolDefault\textlambda{T3}
\DeclareUnicodeCharacter{03BB}{\textlambda}


\begin{document}

\frame[plain]{\titlepage}
\only<article>{\maketitle}

\section{Ein Beispielprogramm}
\only<presentation>{\subsection*{}}

\newcommand{\lab}[1]{\textsuperscript{\scriptsize\sffamily#1}}

\begin{frame}
\frametitle{Umständlich konstant eins}
Unser Beispielprogramm:
\ttfamily
\begin{tabbing}
(\=λ c. \=let \=p = (\=λ k.\=\ k 1) in\=\ p (\=λ x.\=\ p c ))\hspace{6mm}\=\\
\visible<2->{ \>\lab{1} \>\lab{2} \> \>\lab{3} \>\lab{4} \> \lab{5} \> \lab{6} \> \lab{7} \>{\normalfont \textsuperscript{$\leftarrow$ Labels}}}
\end{tabbing}
\end{frame}

\section{Semantik}
\only<presentation>{\subsection*{}}

\newcommand{\Stop}{\text{Stop}}
\begin{frame}
\frametitle{In 10 Schritten zur Eins}
{\ttfamily
\begin{tabbing}
(\=λ c. \=let \=p = (\=λ k.\=\ k 1) in\=\ p (\=λ x.\=\ p c ))\\
\>\lab{\alert<1>{1}} \>\lab{\alert<2>{2}} \> \>\lab{\alert<4,8>{3}} \>\lab{\alert<5,9>{4}} \> \lab{\alert<3>{5}} \> \lab{\alert<6>{6}} \> \lab{\alert<7>{7}}
\end{tabbing}}

\begin{overprint}
\onslide<+>
\begin{block}{Schritt 1}
\begin{itemize}
\item Evaluiere:\phantom{$\langle$}1
\item Parameter: $[\Stop]$
\item Variablenbelegung: \\
$\{\}$
\item Contour-Zähler: \textit0
\end{itemize}
\end{block}

\onslide<+>
\begin{block}{Schritt 2}
\begin{itemize}
\item Evaluiere:\phantom{$\langle$}2
\item Bindungsumgebung: $\{ 1 \mapsto \textit 0 \}$
\item Variablenbelegung: \\
$\{ (c,\textit 0) \mapsto \Stop \}$
\item Contour-Zähler: \textit0
\end{itemize}
\end{block}

\onslide<+>
\begin{block}{Schritt 3}
\begin{itemize}
\item Evaluiere:\phantom{$\langle$}5
\item Bindungsumgebung: $\beta \coloneqq \{ 1 \mapsto \textit 0,\ 2 \mapsto \textit 1 \}$
\item Variablenbelegung: \\
$\{ (c,\textit 0) \mapsto \Stop,\ (p,\textit 1) \mapsto \langle 3,\beta\rangle\}$
\item Contour-Zähler: \textit1
\end{itemize}
\end{block}

\onslide<+>
\begin{block}{Schritt 4}
\begin{itemize}
\item Evaluiere:\phantom{$\langle$}$\langle 3, \beta\rangle$
\item Parameter: $[\langle6, \beta\rangle]$
\item Variablenbelegung: \\
$\{ (c,\textit 0) \mapsto \Stop,\ (p,\textit 1) \mapsto \langle 3 ,\beta\rangle\}$
\item Contour-Zähler: \textit2
\end{itemize}
\end{block}

\onslide<+>
\begin{block}{Schritt 5}
\begin{itemize}
\item Evaluiere:\phantom{$\langle$}4
\item Bindungsumgebung: $\beta \cup \{ 3 \mapsto \textit 2 \}$
\item Variablenbelegung: \\
$\{ (c,\textit 0) \mapsto \Stop,\ (p,\textit 1) \mapsto \langle 3 ,\beta\rangle,\ (k,\textit 2) \mapsto \langle6, \beta\rangle\}$
\item Contour-Zähler: \textit2
\end{itemize}
\end{block}

\onslide<+>
\begin{block}{Schritt 6}
\begin{itemize}
\item Evaluiere:\phantom{$\langle$}$\langle6, \beta\rangle$
\item Parameter: $[\text{\tt 1}]$
\item Variablenbelegung: \\
$\{ (c,\textit 0) \mapsto \Stop,\ (p,\textit 1) \mapsto \langle 3 ,\beta\rangle,\ (k,\textit 2) \mapsto \langle6, \beta\rangle\}$
\item Contour-Zähler: \textit3
\end{itemize}
\end{block}

\onslide<+>
\begin{block}{Schritt 7}
\begin{itemize}
\item Evaluiere:\phantom{$\langle$}7
\item Bindungsumgebung: $\beta \cup \{ 6 \mapsto \textit 3 \}$
\item Variablenbelegung: \\
$\{ (c,\textit 0) \mapsto \Stop,\ (p,\textit 1) \mapsto \langle 3 ,\beta\rangle,\ (k,\textit 2) \mapsto \langle6, \beta\rangle, (x,\textit 3) \mapsto \text{\tt 1}\}$
\item Contour-Zähler: \textit3
\end{itemize}
\end{block}

\onslide<+>
\begin{block}{Schritt 8}
\begin{itemize}
\item Evaluiere:\phantom{$\langle$}$\langle 3, \beta\rangle$
\item Parameter: $[\Stop]$
\item Variablenbelegung: \\
$\{ (c,\textit 0) \mapsto \Stop,\ (p,\textit 1) \mapsto \langle 3 ,\beta\rangle,\ (k,\textit 2) \mapsto \langle6, \beta\rangle, (x,\textit 3) \mapsto \text{\tt 1}\}$
\item Contour-Zähler: \textit4
\end{itemize}
\end{block}

\onslide<+>
\begin{block}{Schritt 9}
\begin{itemize}
\item Evaluiere:\phantom{$\langle$}4
\item Bindungsumgebung: $\beta \cup \{ 3 \mapsto \textit 4 \}$
\item Variablenbelegung: \\
$\{ (c,\textit 0) \mapsto \Stop,\ (p,\textit 1) \mapsto \langle 3 ,\beta\rangle,\ (k,\textit 2) \mapsto \langle6, \beta\rangle, (x,\textit 3) \mapsto \text{\tt 1},$\\$\phantom{\{}(k,\textit 4) \mapsto \Stop\}$
\item Contour-Zähler: \textit4
\end{itemize}
\end{block}

\onslide<+>
\begin{block}{Schritt 10}
\begin{itemize}
\item Evaluiere:\phantom{$\langle$}$\Stop$
\item Parameter: $[1]$
\item Variablenbelegung: \\
$\{ (c,\textit 0) \mapsto \Stop,\ (p,\textit 1) \mapsto \langle 3 ,\beta\rangle,\ (k,\textit 2) \mapsto \langle6, \beta\rangle, (x,\textit 3) \mapsto \text{\tt 1},$\\$\phantom{\{}(k,\textit 4) \mapsto \Stop\}$
\item Contour-Zähler: \textit5
\end{itemize}
\end{block}

\end{overprint}

\end{frame}

\section{Exakte Analyse}
\only<presentation>{\subsection*{}}

\begin{frame}
\frametitle{Halteproblematisch}

{\ttfamily
\begin{tabbing}
(\=λ c. \=let \=p = (\=λ k.\=\ k 1) in\=\ p (\=λ x.\=\ p c ))\\
\>\lab{1} \>\lab{2} \> \>\lab{3} \>\lab{4} \> \lab{5} \> \lab{6} \> \lab{7} 
\end{tabbing}
}

\begin{block}{Tatsächliche Aufrufe}
\begin{align*}
(4, \beta \cup \{ 3 \mapsto \textit 2\}) &\leadsto \langle6, \beta\rangle \\
(4, \beta \cup \{ 3 \mapsto \textit 4\}) &\leadsto \Stop \\
(5, \beta) &\leadsto \langle3, \beta\rangle \\
(7, \beta \cup \{ 6 \mapsto \textit 3 \}) &\leadsto \langle3, \beta\rangle 
\end{align*}
\end{block}
\end{frame}

\section{0CFA}
\only<presentation>{\subsection*{}}

\begin{frame}
\frametitle{Ohne Tiefgang}

{\ttfamily
\begin{tabbing}
(\=λ c. \=let \=p = (\=λ k.\=\ k 1) in\=\ p (\=λ x.\=\ p c ))\\
\>\lab{1} \>\lab{2} \> \>\lab{3} \>\lab{4} \> \lab{5} \> \lab{6} \> \lab{7} 
\end{tabbing}
}

\begin{block}{Prognostizierte Aufrufe}
\begin{align*}
4 &\leadsto 6 \\
4 &\leadsto \Stop \\ 
5 &\leadsto 3 \\
7 &\leadsto 3
\end{align*}
\end{block}
\end{frame}

\section{1CFA}
\only<presentation>{\subsection*{}}


\begin{frame}
\frametitle{Etwas Tiefgang}

{\ttfamily
\begin{tabbing}
(\=λ c. \=let \=p = (\=λ k.\=\ k 1) in\=\ p (\=λ x.\=\ p c ))\\
\>\lab{1} \>\lab{2} \> \>\lab{3} \>\lab{4} \> \lab{5} \> \lab{6} \> \lab{7} 
\end{tabbing}
}

\begin{block}{Prognostizierte Aufrufe}
\begin{align*}
(4, \hat\beta \cup \{ 3 \mapsto 5\}) &\leadsto \langle6, \hat\beta\rangle \\
(4, \hat\beta \cup \{ 3 \mapsto 7\}) &\leadsto \Stop \\
(5, \hat\beta) &\leadsto \langle3, \hat\beta\rangle \\
(7, \hat\beta \cup \{ 6 \mapsto 4 \}) &\leadsto \langle3, \hat\beta\rangle 
\end{align*}
\hfill wobei $\hat\beta \coloneqq \{ 1 \mapsto \text{top},\ 2 \mapsto 1\}$
\end{block}
\end{frame}


\section{Slicing}
\only<presentation>{\subsection*{}}

\begin{frame}
\frametitle{Schnittmuster \only<1>{ohne}\only<2>{mit} Shivers}

{\ttfamily
\begin{tabbing}
(\=λ c. \=let \=p = (\=λ k.\=\ k 1) in\=\ p (\=λ x.\=\ p c ))\\
\>\lab{1} \>\lab{2} \> \>\lab{3} \>\lab{4} \> \lab{5} \> \lab{6} \> \lab{7} 
\end{tabbing}
}
\begin{center}
\hfill
\begin{tikzpicture}[every node/.style={draw}]
\path     node[circle] (1) {1}
-- ++(0,-1) node[trapezium] (2) {2}
-- ++(1,-1) node[circle] (3) {3}
-- ++(0,-1) node[rectangle] (4) {4}
-- ++(-1,1) node[rectangle] (5) {5}
-- ++(2,-1) node[circle] (6) {6}
-- ++(0,-1) node[rectangle] (7) {7}
-- ++(-2,0) node[diamond] (s) {};
\draw[->] (1) -- (2);
\draw[->] (2) -- (5);
\draw[->] (5) -- (3);
\draw[->] (3) -- (4);
\draw[->] (4) -- (6);
\draw[->] (6) -- (7);
\draw[->] (7) to[out=45,in=0] (3);
\draw[->] (4) -- (s);
\visible<1>{
\draw[->] (4) to[out=180,in=-135] (1);
\draw[->] (4) -- (3);
\draw[->] (5) to[out=125,in=-125] (1);
\draw[->] (5) -- (6);
\draw[->] (5) -- (s);
\draw[->] (7) to[out=45,in=0] (1);
\draw[->] (7) -- (6);
\draw[->] (7) -- (s);
}
\end{tikzpicture}
\hfill
\llap{
\begin{tikzpicture}
\matrix[anchor=west]{
\node[draw,circle] {}; \pgfmatrixnextcell \node {Lambda}; \\
\node[draw,trapezium] {}; \pgfmatrixnextcell \node {Let}; \\
\node[draw,rectangle] {}; \pgfmatrixnextcell \node {Aufruf}; \\
\node[draw,diamond] {}; \pgfmatrixnextcell \node {Stop}; \\
};
\end{tikzpicture}
}
\end{center}

\end{frame}


\section{Fazit}
\only<presentation>{\subsection*{}}

\begin{frame}
\frametitle{Fazit}
\begin{block}{}
\begin{itemize}
\item Shivers Algorithmus ist korrekt.
\item Slicing-Framework für funktionale Sprache denkbar.
\item Shivers Algorithmus kann dabei helfen.
\item Es besteht noch Arbeitsbedarf.
\end{itemize}
\end{block}
\end{frame}


\only<presentation>{\section{}\subsection*{}}

\setbeamercolor{normal text}{bg=black}
\setbeamertemplate{background canvas}{}
\frame<presentation>[plain]{}
\end{document}

